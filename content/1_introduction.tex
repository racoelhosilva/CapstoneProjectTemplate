\section{Introduction}
\label{section:introduction}

% Latex Referencing Tips

% For easier referencing throughout the document, I recommend using prefixes:
% - fig
% - section
% - code

% Defining sections without number: \section*{section-name}

% Referencing the glossary: \Gls{label}
% Citing the bibliography: \cite{label}
% Referencing labels with custom name: \hyperref[label]{displayed text}
% Referencing a figure: \Cref{label}

% Figure Guides

% Example with 2 images side by side
%\begin{figure}[ht]
%    \centering
%    \includegraphics[width=0.3\textwidth]{path/image1.png}
%    \hspace{0.07\textwidth}
%    \includegraphics[width=0.6\textwidth]{path/image2.jpg}
%    \caption{Caption of the}
%    \label{fig:graphs}
%\end{figure}

% Code Guides

% Example C code
%\begin{figure}[ht]
%\begin{minted}{c}
%int main() {
%    return 0;
%}
%\end{minted}
%\caption{Example formatted code block}
%\label{code:example}
%\end{figure}

\subsection{Background and Scope}

\begin{itemize}
    \item Introduce the goal of the report
    \item Present the organizational context of the project/internship (company, institution, investigation center/lab, etc.).
    \item Present the given problem and motivation for the work developed.
\end{itemize}

\subsection{Goals and Expected Results}

Point out the goals/objectives of the work and the expected results.

\subsection{Report Structure}

Explain the structure of the LaTeX report.

The report is structured into the following sections:

\begin{itemize}
    \item \hyperref[section:introduction]{\textbf{1. Introduction}}: 
    \item \hyperref[section:methodology]{\textbf{2. Used Methodology and Main Activities Developed}}: 
    \item \hyperref[section:development]{\textbf{3. Solution Development}}: 
    \item \hyperref[section:conclusion]{\textbf{4. Conclusion}}:
\end{itemize}